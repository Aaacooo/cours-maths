\documentclass[12pt,a4paper,french]{book}
\usepackage{graphicx} % Required for inserting images
\usepackage{multirow}
\usepackage[frenchb]{babel}
\usepackage{fancybox,framed}
\usepackage{amssymb}
\usepackage{amsmath}
\usepackage{array}
\usepackage{setspace}
\usepackage{enumitem}
\usepackage{titlesec}
\usepackage{fancybox}
\usepackage{mathrsfs}
\usepackage{stmaryrd}
\title{Cours}
\author{C. LACOUTURE}
\date{Année scolaire 2024-2025, MPSI2, Lycée Carnot}

% Gérer les espacements %
\setlength{\parskip}{4pt}
\titlespacing*{\chapter}{0pt}{3pt}{2pt}
\titlespacing*{\section}{0pt}{2pt}{1pt}
\titlespacing*{\subsection}{0pt}{3pt}{0pt}
\titlespacing*{\subsubsection}{0pt}{2pt}{0pt}

\begin{document}
\maketitle
\tableofcontents
	
\part{Algèbre}
\chapter{Structure de groupe}
	\begin{spacing}{1.0}
	\section{Présentation}
		\subsection{Exemple préliminaire}
		\begin{spacing}{1.0}
			L'ensemble \(\mathbb{Z}\) pour l'addition \(+\) est tel que:
			\begin{enumerate}[itemsep=2pt, topsep=1pt, partopsep=0pt, parsep=1pt]
				\item \(\forall x, y \in \mathbb{Z}\)
				\item \(\forall x, y, z \in \mathbb{Z}, (x + y) + z = x + (y + z)\)
				\item \(\forall x \in \mathbb{Z}, x + 0 = 0 + x = x\)
				\item \(\forall x \in \mathbb{Z}, x + (-x) = (-x) + x = 0\)
				\item et de plus \(\forall x, y \in \mathbb{Z}, x + y = y + x\)
			\end{enumerate}
			Ainsi \((\mathbb{Z}, +)\) est un groupe abélien / groupe commutatif.
		\end{spacing}
		\subsection{Définition générale}
		\begin{spacing}{1.0}
			Soit un ensemble \( G \) muni d'une loi \( * \). Dès lors, \( (G, *) \) a une structure de groupe si et seulement si :
			
			\boxed{
				\begin{minipage}{\textwidth}
					\begin{enumerate}[itemsep=2pt, topsep=1pt, partopsep=0pt, parsep=1pt]
						\item \( * \) est une Loi de Composition Interne (LCI) sur \( G \). C'est-à-dire \(\forall x, y \in G, x * y \in G\).
						
						\item \( * \) est associative. C'est-à-dire \(\forall x, y, z \in G, (x * y) * z = x * (y * z)\).
						
						\item \( G \) a un élément neutre \( e \) pour \( * \). \(\exists e \in G\) tel que \(\forall x \in G, x * e = e * x = x\).
						
						\item Tout élément de \( G \) a un symétrique dans \( G \). \(\forall x \in G, \exists x' \in G\) tel que \(x * x' = x' * x = e\).
					\end{enumerate}
				\end{minipage}
			}
			
			Si, de plus, \( * \) est commutative sur \( G \), c'est-à-dire \(\forall x, y \in G, x * y = y * x\), alors \( G \) est un groupe commutatif (ou abélien).
			
			\textbf{Remarques concernant la définition :}
			 
			a) attention à la place des quantificateurs : pour l'élément neutre (3.) c'est \(\exists\) puis \(\forall\) et pour le symétrique c'est \(\forall\) puis \(\exists\).
			
			b) attention aux 2 égalités dans la définition de l'élément neutre et des symétriques d'un élément car * ne commute pas forcément.
		\end{spacing}
		\subsection{Exemples usuels}
			\subsubsection{Ensembles de nombres}
			\((\mathbb{Z}, +), (\mathbb{Q}, +), (\mathbb{R}, +), (\mathbb{C}, +), (\mathbb{Q}^*, +), (\mathbb{R}^*_+, \times), (\mathbb{R}^*, \times), (\mathbb{C}^*, \times)\)
			Sont tous des groupes commutatifs.
			\subsubsection{Ensemble des bijections}
			\begin{spacing}{1.0}
			Soit E un ensemble et \(\mathscr{B}(E)\) l'ensemble des bijections de E vers E.
			\begin{center}
				\boxed{
					(\mathscr{B}(E), \circ) \text{ est un groupe \underline{non} commutatif.}
					}
			\end{center}
			Le neutre pour \(\circ\) est  : \(\operatorname{Id}_E\) et le symétrique de f pour \(\circ\) est \(f^{-1}\)
			\end{spacing}
			\subsubsection{Ensemble des parties}
			\begin{spacing}{1.0}
				Soit E un ensemble et \(\mathscr{P}(E)\) l'ensemble des parties de E.
				\begin{center}
					\boxed{
						(\mathscr{P}(E), \Delta) \text{ est un groupe commutatif.}
					}
				\end{center}
				Le neutre pour \(\Delta\) est  : \(\emptyset\) car \(\forall A \subset E : A \Delta \emptyset = \emptyset \Delta A = A\)
				Et le symétrique de A pour \(\Delta\) est A car \(A \Delta A = \emptyset\)
			\end{spacing}
		\subsection{Compléments}
			\subsubsection{Unicité}
			Sont uniques l'élément neutre et le symétrique de tout élément.
			\subsubsection{Formules concernant le symétrique}
			\textbf{a) \(\forall x,y \in G\) :}
			\((x * y)' = y' * x'\) (avec ' pour symétrique)
			Attention à l'ordre car le groupe n'est pas forcément commutatif.
			\textbf{b)} \(\forall x \in G (x')' = x\)
			\subsubsection{Régularité de tout élément}
			\(\forall x, y, z \in G \) : \boxed{x*z = y*z \implies x = y \) et \(z*x =z*y \implies x = y}
			
			Ainsi on traduit que tout élément z est régulier dans le groupe (G, *) c'est à dire dans un groupe, on peut "simplifier par tout élément".
			\subsubsection{Plus généralement : résolution d'équation}
			\(\forall x, y, z \in G : x * y = z\)
			
			\(\implies x' * (x * y) = x' * z\)
			
			\(\implies (x' * x) * y = x' * z\) car * est associative
			
			\(\implies e * y = x' * z\) car x' est symétrique de x pour *
			
			\(\implies y = x' * z\) car e neutre pour *
			
			Dans un groupe : tout élément peut, dans une égalité, "passer dans l'autre membre" sous la forme de son symétrique ... en tenant compte de l'ordre car * ne commute pas forcément.
		\subsection{Notations}
		En pratique, un groupe est noté : 
		(G, +), même si + n'est pas l'addition classique (notation \underline{additive}) ou (G, \(\cdot\)), même si \(\cdot\) n'est pas la multiplication classique (notation \underline{multiplicative})
		On définit alors : 
			\subsubsection{En notation additive}
			\(\forall x \in G, \forall n \in \mathbb{Z}\)
			
			\begin{center}
				\boxed{%
					\begin{minipage}{0.9\textwidth}
						\begin{enumerate}[itemsep=2pt, topsep=1pt, partopsep=0pt, parsep=1pt]
							\item \underline{si \(n = 0\)} : \(0x = 0_G\) neutre de \(G\) pour \(+\).
							\item \underline{si \(n \in \mathbb{N}^*\)} : \(nx = x + x + \dots + x\) (n fois).
							\item \underline{si \(n \in \mathbb{Z}_-^*\)} : \(nx = (-x) + \dots + (-x)\) (-n fois) où \(-x\) est le symétrique de \(x\) pour \(+\).
						\end{enumerate}
					\end{minipage}%
				}
			\end{center}
			\subsubsection{En notation multiplicative}
			Les puissances d'un selon : \(\forall x \in G, \forall n \in \mathbb{Z}\)
			
			\begin{center}
				\boxed{
					\begin{minipage}{0.9\textwidth}
						\begin{enumerate}[itemsep=2pt, topsep=1pt, partopsep=0pt, parsep=1pt]
							\item \underline{si \(n = 0\)} : \(x^0 = 1_G\) neutre de \(G\) pour \(\cdot\).
							\item \underline{si \(n \in \mathbb{N}^*\)} : \(x^n = x \cdot x \cdot \dots \cdot x\) (n fois).
							\item \underline{si \(n \in \mathbb{Z}_-^*\)} : \(x^n = (x^{-1}) \cdot \dots \cdot (x^{-1})\) (-n fois) où \(x^{-1}\) est le symétrique de \(x\) pour \(\cdot\).
						\end{enumerate}
					\end{minipage}
				}
			\end{center}
			\subsubsection{Propriétés des multiples d'un élément}
			\(\forall x \in G, \forall m, n \in \mathbb{Z}\)
			
			\begin{center}
				\boxed{%
					\begin{minipage}{0.9\textwidth}
						\begin{enumerate}[itemsep=2pt, topsep=1pt, partopsep=0pt, parsep=1pt]
							\item \((m + n) x = m x + n x\) \quad (additivité classique dans \(\mathbb{Z}\) qui devient une loi de groupe sur \(G\))
							\item \(-(nx) = (-n)x = n(-x)\) \quad (-n est l'opposé classique dans \(\mathbb{Z}\) et ($-x$) symétrique de $x$ dans $G$)
							\item \(m (n x) = (m \times n) x\) \quad (\(\times\) produit classique dans \(\mathbb{Z}\) et loi de groupe sur \(G\))
						\end{enumerate}
					\end{minipage}
				}
			\end{center}
			\subsubsection{Propriétés des puissances d'un élément}
			\(\forall x \in G, \forall m, n \in \mathbb{Z}\)
			\begin{center}
				\boxed{
					\begin{minipage}{0.9\textwidth}
						\begin{enumerate}[itemsep=2pt, topsep=1pt, partopsep=0pt, parsep=1pt]
							\item \(x^{m+n} = x^m \cdot x^n\)
							\item \((x^m)^{-1} = (x^{-1})^m = x^{-m}\)
							\item \(x^{mn} = (x^m)^n = (x^n)^m\)
						\end{enumerate}
					\end{minipage}
				}
			\end{center}
		\subsection{Autres remarques}
			\subsubsection{Concernant les propriétés des puissances ou des multiples précédents}
			Parmis toute les propriétés citées, on a pas cité \((x \cdot y)^n = x^n \cdot y^n\) car c'est faux si \(\cdot\) ne commute pas forcément. 
			\((x \cdot y)^n = (x \cdot y) \cdot (x \cdot y) \cdot\) ...(n fois) \(\cdot (x \cdot y)\)  \(\neq x^n \cdot y^n = x \cdot\) ...(n fois) \(\cdot x \cdot y \cdot\) ...(n fois) \(\cdot y \)
			\subsubsection{Concernant les produits cartésien de groupes}
			Soit (G, +) et (G', \(\cdot\)) 2 groupes.
			G \(\times\) G' est un groupe pour * tel que (x, x')*(y, y') = (x + y, x' \(\cdot\) y').
			
			Le neutre de G \(\times\) G' pour * est : (\(0_G, 1_{G'}\)).
			
			Le symétrique de (x, x') dans G \(\times\) G' pour * est : \((-x, x'^{-1})\)
	\section{Sous-groupes}
		\subsection{Définition}
		Soit (\(G, \cdot\)) un groupe et \(H \subset G\), H est un sous-groupe de G pour \(\cdot\) si et seulement si la restriction de \(\cdot\) avec les éléments de H est muni d'une structure de groupe
		\subsection{Caractérisations}
		\begin{center}
			\boxed{%
				\begin{minipage}{0.9\textwidth}
					\(H\) sous-groupe de \(G\) pour \(\cdot\) : 
					
					\[
					\Leftrightarrow
					\]
					\begin{itemize}
						\item \(1_G \in H\)
						\item \(\forall x, y \in H, \quad x \cdot y \in H\)
						\item \(\forall x \in H, \quad x^{-1} \in H\)
					\end{itemize}
					
					\[
					\Leftrightarrow
					\]
					\begin{itemize}
						\item \(1_G \in H\)
						\item \(\forall x, y \in H, \quad x \cdot y^{-1} \in H\)
					\end{itemize}
				\end{minipage}
			}
		\end{center}
		
		\textbf{Remarque : } Dans la majorité des cas dans la suite on montrera que \((H, \cdot)\) est un groupe en montrant que c'est un sous-groupe d'un groupe usuel.
		\subsection{Exemples usuels}
			\subsubsection{Exemple général}
			Les puissances d'un élément.
			
			\boxed{%
				\text{Soit } (G, \cdot) \text{ un groupe et } a \in G, \quad
				H = \{ a^z \mid z \in \mathbb{Z} \}
				\text{ est un sous-groupe de } G \text{ pour } \cdot.
			}
			
			Les multiples d'un élément.
			
			\boxed{%
				\text{Soit } (G, +) \text{ un groupe et } a \in G, \quad
				H = \{ za \mid z \in \mathbb{Z} \}
				\text{ est un sous-groupe de } G \text{ pour } +.
			}
			\subsubsection{Exemples particuliers}
			\boxed{%
				(\mathbb{U}, \times) \text{est un groupe.}
			}
			Car sous-groupe de \((\mathbb{C}^*, \times)\)
			
			\boxed{%
				(\mathbb{U}n, \times) \text{est un groupe.}
			}
			Car sous-groupe de \((\mathbb{C}^*, \times)\) 
			\subsubsection{Sous-groupes de $(\mathbb{Z},+)$}
			\boxed{
				\text{Les sous-groupes de } \mathbb{Z} \text{ pour + sont les ensembles de la forme } n\mathbb{Z} = {nz \mid z \in \mathbb{Z}}
			}
		\subsection{Propriétés}
		\subsubsection{Intersection}
		"Toute intersection de sous-groupes est un sous-groupe."
		
		\boxed{%
			\begin{minipage}{0.95\textwidth}
				Soit \((G, \cdot)\) un groupe.  
				\((H_i)_{i \in \Delta}\) une famille de sous-groupes de \(G\) pour \(\cdot\).  
				Alors \(H = \bigcap_{i \in \Delta} H_i\) est un sous-groupe de \(G\) pour \(\cdot\).
			\end{minipage}
		}
			\subsubsection{Faux pour la réunion $\cup$}
			\textbf{a) contre-exemple : dans (Z, +)}
			
			Soit \(H_1 = 2\mathbb{Z} = \{ 2z \mid z \in \mathbb{Z} \}\) et \(H_2 = 3\mathbb{Z} = \{ 3z \mid z \in \mathbb{Z} \}\). 
			
			Mais H = \(2\mathbb{Z} \cup 3\mathbb{Z}\) n'est pas un sous-groupe de \(\mathbb{2}\) pour + car :
			
			\(2 \in 2\mathbb{Z} \subset 2\mathbb{Z} \cup 3\mathbb{Z} = H\) et \(3 \in 3\mathbb{Z} \subset 2\mathbb{Z} \cup 3\mathbb{Z} = H\)
			
			Mais \(2+3 = 5 \notin H\)
			
			\textbf{b) "sous conditions"}
			
			Soit \(G_1, G_2\) sous groupes de G pour \(\cdot\).
			
			\boxed{
				G_1 \cup G_2 \text{ est un sous-groupe de G pour } \cdot
				\Leftrightarrow G_1 \subset G_2 \text{ ou } G_2 \subset G_1
			}
	\section{Morphismes de groupes}
		\subsection{Définition}
		Soit (G, +) et (G', $\cdot$) 2 groupes, soit f : G $\rightarrow$ G'.
		
		\boxed{
			\text{f est un } \underline{morphisme} \text{ de groupes si et seulement si } \forall x,y \in G  : f(x+y) = f(x) \cdot f(y) 
		}
		
		De plus :
		
		\textbf{a)} si G = G' et + = $\cdot$ : f est un endomorphisme
		
		\textbf{b)} si f est bijective : f est un isomorphisme
		
		\textbf{b)} si G = G' et + = $\cdot$ et f bijective : f est un automorphisme
		
		\subsection{Exemples usuels}
		exp est un isomorphisme de ($\mathbb{R}$, +) vers ($\mathbb{R}^*_+$, +)
		
		ln est un isomorphisme de ($\mathbb{R}^*_+$, +) vers ($\mathbb{R}$, +)
		
		exp est un morphisme de ($\mathbb{C}$, +) vers ($\mathbb{C}^* $, $\times$) car exp n'est pas injective sur $\mathbb{C}$
		\subsection{Propriétés}
		Soit $f : (G, +) \rightarrow (G', \cdot)$ un morphisme
		
		\subsubsection{Images du neutre et du symétrique de tout élément}
		\textbf{a)} \boxed{f(0_G) = 1_{G'}}
		
		\textbf{b)} \boxed{\forall x \in G : f(-x) = (f(x))^{-1}}
		Ainsi on retrouve : $e^{-x} = \frac{1}{e^x}$
		
		\subsubsection{Caractérisation de l'injectivité}
		\textbf{a) notations}
		
			\underline{le noyau de f est :} \boxed{Ker(f) = {x \in G \mid f(x) = 1_{G'}}}
			
			\underline{l'image de f est :} \boxed{Im(f) = {f(x) \mid x \in G}}
			
		\textbf{b) caractérisation}
		
			\boxed{\text{f est injectivite} \Leftrightarrow Ker(f) = \{0_G\}}
		
		\subsubsection{Image (réciproque) d'un sous-groupe}
		\textbf{a) image :}
		
		\boxed{
			\text{Soit H un sous-groupe de G pour + alors f(H) est un sous-groupe de G' pour } \cdot
		} 
		
		\textbf{b) image réciproque:}
		
		\boxed{
			\text{Soit H un sous-groupe de G' pour } \cdot \text{alors f(H) est un sous-groupe de G' pour + }
		} 
		
		\underline{c) conséquences :}
		
		\boxed{
			Im(f) = f(G) \text{ est un sous-groupe de G' pour } \cdot
		}
		
		\boxed{
			Ker(f) = f_{rec}(\{1_{G'}\} ) \text{ est un sous groupe de G pour +}
		}
		
\chapter{Structure d'anneau et de corps}
	\section{Structure d'anneau}
		\subsection{Présentation}
			\subsubsection{Exemple préliminaire}
			\subsubsection{Définition générale}
			\subsubsection{Notations}
			\subsubsection{Intégrité}
			\subsubsection{Exemples usuels}
		\subsection{Propriétés}
			\subsubsection{Élément absorbant}
			\subsubsection{Ensemble des inversibles}
			\subsubsection{"Opposé" d'un produit}
			\subsubsection{Loi "soustraction"}
			\subsubsection{Formule du binôme de Newton}
			\subsubsection{Formule de factorisation}
		\subsection{Sous-anneau}
			\subsubsection{Caractérisation}
			\subsubsection{Exemple usuel : sous-anneau des décimaux}
	\section{Structure de corps}
		\subsection{Définition}
		\subsection{Exemples usuels}
		\subsection{Propriétés}
			\subsubsection{Intégrité}
			\subsubsection{Commutativité}
		\subsection{Sous-corps}
	\end{spacing}
	
	
	
\chapter{Corps des nombres réels}
	\section{Généralités}
	\section{Borne supérieure ou inférieure d'une partie de $\mathbb{R}$}
		\subsection{Définition}
		\subsection{Existence-unicité}
			\subsubsection{Existence}
			\subsubsection{Unicité}
		\subsection{Mise en garde}
		\subsection{Caractérisation}
	\section{Valeurs approchées d'un réel à $\alpha$ près (où $\alpha \in \mathbb{Q}^{\ast+}$)}
		\subsection{Résultat et définition}
		\subsection{Cas où $\alpha = 1$}
		\subsection{Cas où $\alpha = \frac{1}{10^{n}} (n \in \mathbb{N})$}
			\subsubsection{Énoncé}
			\subsubsection{Convergence}
	\section{Densité}
		\subsection{Définitions}
			\subsubsection{Intervalle}
			\subsubsection{Densité}
		\subsection{Caractérisation}
			\subsubsection{Générale}
			\subsubsection{Plus précisément}
			\subsubsection{Remarques}
			\subsubsection{Exemple usuel}
		\subsection{Compléments}
			\subsubsection{Concernant $\mathbb{Q}$}
			\subsubsection{Concernant$\bar{\mathbb{Q}}$ (complément de $\mathbb{Q}$ dans $\mathbb{R}$)}
			
			
\chapter{Corps des nombres complexes}
	\section{Conjugaison}
		\subsection{Définition}
		\subsection{Propriétés}
			\subsubsection{Formules}
			$\forall z \in \mathbb{C}, \ z+ \bar{z} = 2\mbox{Re}(z), \ z - \bar{z} = 2i\mbox{Im}(z)$
		
			\subsubsection{Caractérisation}
				\begin{equation} \begin{split}
						z \in \mathbb{R} &\Leftrightarrow z = \bar{z} \\
						z \in i\mathbb{R} &\Leftrightarrow z = -\bar{z}
					\end{split}
				\end{equation} 
			\subsubsection{Pratique}
				Quand un nombre complexe est écrit au dénominateur, on le multiplie par son conjugué.
	\section{Module}
		\subsection{Définition}
			$\forall z \in \mathbb{C}$, le module de z est $\left| z\right| = \sqrt{z \bar{z}}$
			\subsubsection{Pratique}
			Pour $z=a+ib$, on a $\left| z\right| = \sqrt{a^2 + b^2}$
			\subsubsection{Lien avec la valeur absolue}
			Le module dans $\mathbb{C}$ prolonge la valeur absolue dans $\mathbb{R}$.
		\subsection{Propriétés}
			\subsubsection{Diverses}
			\[\begin{split}
				\forall z \in \mathbb{C},\left| z\right| \geqslant 0 &\mbox{ et } \left| z\right| = 0 \Leftrightarrow z= 0 \\
				\left| z \right| &= \left| \bar{z} \right| \\
				\forall z,z' \in \mathbb{C},\left| zz' \right|& = \left| z \right| \left| z' \right| 
			\end{split} \]
			\subsubsection{(Double) inégalité triangulaire}
			\[\forall z,z' \in \mathbb{C},\ \left|\left|z\right|-\left|z'\right|\right|\leqslant\left|z+z'\right| \leqslant \left|z\right| + \left|z'\right|\]
		\subsection{Nombres complexes de module 1}
			\subsubsection{Description}
			$U = \left\{z \in \mathbb{C}, \ \left|z\right| = 1\right\}$. Les complexes de module 1 s'écrivent $z = \cos(\theta)+i\sin(\theta)$, ce qu'on note : $ e^{i\theta}$ où $\theta \in \mathbb{R} \cdot\setminus 2\pi \mathbb{Z}$
			\subsubsection{Remarque sur l'écriture $e^{i\theta}$}
			$\forall \theta, \ e^{-i\theta} = \bar{e^{i\theta}} = \frac{1}{e^{i\theta}} $
			\subsubsection{Produit}
			$\forall \theta, \theta', \ e^{i\theta} e^{i\theta'} = e^{i(\theta+\theta')}$
			\subsubsection{Formule de Moivre}
			$\forall n \in \mathbb{Z}, \theta \in \mathbb{R},\ (\cos(\theta)+i\sin(\theta))^n = \cos(n\theta)+i\sin(n\theta)$
			\subsubsection{Formules à savoir}
				$\forall \theta \in \mathbb{R}$ , $\left\{ \begin{array}{ll}
				\cos(\theta) = \frac{e^{i\theta + e^{-i\theta}}}{2} \\
				\sin(\theta) = \frac{e^{i\theta}- e^{-i \theta}}{2i}
			\end{array} \right. $
			
			$\forall \theta \in \mathbb{R}$ , $\left\{ \begin{array}{ll}
				1+e^{i \theta} = 2\cos\left(\frac{\theta}{2}\right) e^{i \frac{\theta}{2} } \\
				1-e^{i \theta} = -2i\sin\left(\frac{\theta}{2}\right) e^{i \frac{\theta}{2} }
			\end{array} \right. $
	\section{Forme trigonométrique}
		\subsection{Définition}
			\subsubsection{Résultat préliminaire}
			$\forall z \in \mathbb{C}^\ast, \exists! \rho > 0, \exists! u \in U \mbox{ tel que } z = \rho u$
			\subsubsection{Conséquence}
			$\forall z \in \mathbb{C}^\ast, \exists! \rho >0, \exists! \theta \in \mathbb{R} \setminus 2\pi \mathbb{Z} \mbox{ tel que } z = \rho e^{i\theta}$
		\subsection{Premiers exemples}
			\subsubsection{Divers}
			\begin{equation*}
				\begin{split}
					1+i &= \sqrt{2}e^{i\frac{\pi}{4}} \\
					1-i &= \sqrt{2}e^{-i\frac{\pi}{4}} \\
					-1+i &= \sqrt{2}e^{i\frac{3\pi}{4}} \\
					-1-i &= \sqrt{2}e^{-i\frac{5\pi}{4}} \\
				\end{split}
			\end{equation*}
			\subsubsection{Caractérisations}
			$\left\{\begin{array}{ll}
				z \in \mathbb{R}^\ast \Leftrightarrow \arg{z} = 0 (\pi)\\
				z \in i\mathbb{R}^\ast \Leftrightarrow \arg{z} = \frac{\pi}{2} (\pi)
			\end{array}\right.$
		\subsection{Relations entre forme algébrique et trigonométrique}
		Soit $z \in \mathbb{C}^\ast$. $z = x + iy = \rho e^{i\theta}$.
		
		Quelles relations a-t-on entre $x,y,\rho,\theta$ ?
			\subsubsection{Sens direct}
			$\begin{array}{ll}
				x = \rho \cos(\theta) \\
				y = \rho \sin(\theta)
			\end{array}$
			\subsubsection{Sens réciproque}
			\begin{equation*}
				\rho = \sqrt{x^2 + y^2}
			\end{equation*}
			\begin{equation*}
				\theta = \left\{ \begin{array}{ll}
					\frac{\pi}{2} &\mbox{ si } x = 0 y > 0\\
					-\frac{\pi}{2} &\mbox{ si } x = 0, y < 0\\
					\arctan\left(\frac{y}{x}\right) &\mbox{ si } x > 0\\
					\pi + \arctan\left(\frac{y}{x}\right) &\mbox{ si } x < 0
				\end{array}\right.
			\end{equation*}
		\subsection{Formules diverses}
		Soient $z_1,z_2 \in \mathbb{C}^\ast$.
		\[\operatorname{arg}(z_1 z_2) = \operatorname{arg}(z_1)+\operatorname{arg}(z_2)\]
		\[\operatorname{arg}\left(\frac{z_1}{z_2}\right) = \operatorname{arg}(z_1)-\operatorname{arg}(z_2)\]
		\subsection{Interprétation géométrique}
	\section{Équation $z^{n}=a$ (où $n\in \mathbb{N}^{\ast},a\in \mathbb{C}^{\ast}$)}
		\subsection{Résolution}
		$z = \left|a\right|^{\frac{1}{n}} e^{i\left(\frac{\operatorname{arg}(a)}{n} + \frac{2k\pi}{n}\right)}$ où $k \in  \llbracket 0,n-1 \rrbracket$
		\subsection{1\up{er} cas particulier : racines n\up{èmes} de l'unité}
			\subsubsection{Définition}
			Pour $n \in \mathbb{N}$, l'ensemble des racines n\up{èmes} de l'unité est :
			\[U_n = \left\{z \in \mathbb{C}, \ z^n = 1\right\}\]
			\subsubsection{Description}
			\[U_n = \left\{e^{\frac{2ik\pi}{n}},\ k \in \llbracket 0,n-1 \rrbracket \right\}\]
			\subsubsection{Propriétés}
			\begin{itemize}
				\item Somme : la somme des n racines n\up{èmes} de l'unité vaut 0. Ainsi, penser aux racines n\up{èmes} de l'unité quand on a l'expression $1 + x + x^2 + ... + x^{n-1}$.
				\item Conjugaison : les racines n\up{èmes} de l'unité sont conjuguées deux à deux.
			\end{itemize}
			
			\subsubsection{Interprétation géométrique}
			\subsubsection{Cas $n=3$}
			Les 3 racines cubiques de 1 dans $\mathbb{C}$ sont $1,\ j = -\frac{1}{2}+i\frac{\sqrt{3}}{2},\ j^2 = -\frac{1}{2}-i\frac{\sqrt{3}}{2}$
			
			Penser donc à $j$ quand on a l'expression $1+x+x^2$.
		\subsection{Cas particulier des racines carrées d'un complexe}
			\subsubsection{Énoncé}
			\subsubsection{Obtention pratique}
			\subsubsection{Équation de d°2 dans $\mathbb{C}$}
	\section{Traduction complexe de transformations géométriques}
		\subsection{Symétries}
			\subsubsection{Par rapport à $O\overrightarrow{i}$}
			\subsubsection{Centrale par rapport à $O$}
			\subsubsection{Par rapport à $O\overrightarrow{j}$}
		\subsection{Translations}
		\subsection{Homothéties}
			\subsubsection{Définition}
			\subsubsection{Traduction complexe}
		\subsection{Rotations}
			\subsubsection{Définition}
			\subsubsection{Traduction complexe}
		\subsection{Similitudes directes}
			\subsubsection{Définition}
			\subsubsection{Traduction complexe}
	\section{Exponentielle complexe}
		\subsection{Définition}
		Soit $z = x+iy \in \mathbb{C}.$ L'exponentielle de $z$ est :
		\[e^{z} = e^{x}e^{iy} = e^{x}(\cos(y)+i\sin(y))\] 
		\subsection{Propriétés}
			\subsubsection{Module - argument}
			L'écriture $e^{x}e^{iy}$ est une forme trigonométrique.
			\subsubsection{Formule fondamentale}
			$\forall z,z' \in \mathbb{C}, \ e^{z+z'} = e^{z}e^{z'}$
			\subsubsection{Résolution de l'équation $e^{z} = a$}
			$e^{z} = a$ a une infinité de solutions : \[z = \ln(\left|a\right|) + i(\arg(a)+2k\pi), k\in \mathbb{Z} \]
			\subsubsection{Égalité}
			\[\forall z,z' \in \mathbb{C}, e^{z} = e^{z'} \Leftrightarrow z-z' \in 2i\pi\mathbb{Z}\]
			
		
\chapter{Anneau $\mathbb{K}[X]$ des polynômes à une indéterminée à coefficients dans un corps $\mathbb{K}$}
	\section{Présentation}
		\subsection{Définitions}
		\subsection{Opérations sur les polynômes}
			\subsubsection{Somme}
			\subsubsection{Multiplication par un élément de $\mathbb{K}$}
			\subsubsection{Multiplication}
		\subsection{Propriétés}
			\subsubsection{Pour la multiplication}
			\subsubsection{Pour la somme}
		\subsection{Structures}
			\subsubsection{Neutres}
			\subsubsection{Intégrité}
			\subsubsection{Inversibles}
		\subsection{Composée}
			\subsubsection{Définition}
			\subsubsection{Degré}
	\section{Division euclidienne dans $\mathbb{K}[X]$}
		\subsection{Énoncé}
			\subsubsection{Unicité}
			\subsubsection{Existence}
		\subsection{Exemples}
		\subsection{Divisibilité}
			\subsubsection{Définition}
			\subsubsection{Divisibilité par $X-a$}
	\section{PGCD,PPCM dans $\mathbb{K}[X]$}
		\subsection{Définition pour PGCD}
			\subsubsection{Remarques préliminaires}
			\subsubsection{Algorithmique}
			\subsubsection{Définition générale}
		\subsection{Propriétés}
			\subsubsection{Relation de Bézout}
			\subsubsection{Formule de factorisation}
		\subsection{Polynômes premiers entre eux}
			\subsubsection{Définition}
			\subsubsection{Caractérisation : théorème de Bézout}
			\subsubsection{Propriétés}
		\subsection{PPCM dans $\mathbb{K}[X]$}
			\subsubsection{Données du problème}
			\subsubsection{Résolution}
			\subsubsection{Conclusion}
	\section{Zéros (ou racines) d'un polynôme}
		\subsection{Définitions}
		\subsection{Relation entre les racines et le degré d'un polynôme}
			\subsubsection{Résultat}
			\subsubsection{Conséquence}
		\subsection{Polynôme dérivé}
			\subsubsection{Définition}
			\subsubsection{Formule de Taylor}
		\subsection{Caractérisation d'un zéro d'ordre $n$}
	\section{Polynômes irréductibles}
		\subsection{Présentation}
			\subsubsection{Définitions}
			\subsubsection{Remarques}
			\subsubsection{Attention au corps considéré}
		\subsection{Décomposition générale}
			\subsubsection{Résultat préliminaire}
			\subsubsection{Existence}
			\subsubsection{Unicité}
			\subsubsection{Première application}
		\subsection{Dans $\mathbb{C}[X]$}
			\subsubsection{Résultat admis}
			\subsubsection{Précision}
			\subsubsection{Décomposition en facteurs irréductibles unitaires dans $\mathbb{C}[X]$}
			\subsubsection{Conséquences}
			\subsubsection{Pratique}
			\subsubsection{Exemple usuel}
		\subsection{Dans $\mathbb{R}[X]$}
			\subsubsection{Polynôme conjugué}
			\subsubsection{Conséquences}
			\subsubsection{Ainsi :}
		\subsection{Pratique de la décomposition en facteurs irréductibles dans $\mathbb{R}[X]$}
			\subsubsection{Utiliser les racines}
			\subsubsection{Utiliser des identités remarquables}
			\subsubsection{Utiliser la méthode du bicarré}
	\section{Relations coefficients-racines}
		\subsection{Données du problème}
			\subsubsection{Rappel}
			\subsubsection{But : généraliser}
			\subsubsection{Définitions et notations}
			\subsubsection{But}
		\subsection{Résolution}
		\subsection{Appplications}
		
\chapter{Fractions rationnelles}
	\section{Présentation}
		\subsection{Définition}
		\subsection{Opérations}
			\subsubsection{Somme}
			\subsubsection{Produit}
			\subsubsection{Structure}
		\subsection{Forme irréductible}
	\section{Décomposition en éléments simples de $F = \frac{A}{B}$ (irréductible)}
		\subsection{Première étape : partie entière}
			\subsubsection{Énoncé}
			\subsubsection{Démonstration par Analyse-Synthèse}
		\subsection{Deuxième étape : décomposition de $\frac{R}{B}$}
			\subsubsection{Énoncé}
			\subsubsection{Démonstration}
		\subsection{Troisième étape : généralisation}
		\subsection{Conséquence}
		\subsection{Quatrième étape : décomposition de $\frac{R}{P^{\alpha}}$}
			\subsubsection{Résultat général}
			\subsubsection{Démonstration}
		\subsection{Conclusion}
	\section{Décomposition dans $\mathbb{C}(X)$}
		\subsection{Forme a priori}
		\subsection{Détermination pratique des $\lambda,\mu$}
		\subsection{Exemple usuel particulier}
		\subsection{Exemple usuel général}
	\section{Dans $\mathbb{R}(X)$}
		\subsection{Forme a priori}
		\subsection{Détermination pratique des $\lambda,\alpha,\beta$}
		\subsection{Exemple usuel}
	\section{Application principale : calculs de primitive de fonctions rationelles}
		\subsection{Définition}
		\subsection{Méthode pour primitiver $f(x)=\frac{P(x)}{Q(x)} \in \mathbb{R}(x)$}
		
\chapter{Groupe symétrique}
	\section{Présentation}
		\subsection{Définitions}
			\subsubsection{Permutation}
			\subsubsection{Groupe symétrique}
			\subsubsection{Cardinal}
		\subsection{Exemples}
			\subsubsection{Généraux}
			\subsubsection{Particulier}
	\section{Éléments générateurs}
		\subsection{Transpositions}
			\subsubsection{Énoncé}
			\subsubsection{Exemples}
		\subsection{Cycles à supports disjoints}
			\subsubsection{Résultat admis}
			\subsubsection{Exemple}
			\subsubsection{Pratique}
	\section{Signature d'une permutation}
		\subsection{Inversions}
		\subsection{Définitions}
		\subsection{Cas d'une transposition}
		\subsection{Cas d'un cycle}
		\subsection{Morphisme signature}
			





\end{document}