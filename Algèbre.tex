\documentclass[12pt,a4paper,french]{book}
\usepackage{graphicx} % Required for inserting images
\usepackage{multirow}
\usepackage[frenchb]{babel}
\usepackage{fancybox,framed}
\usepackage{amssymb}
\usepackage{amsmath}
\usepackage{array}
\title{Cours}
\author{C. LACOUTURE}
\date{Année scolaire 2024-2025, MPSI2, Lycée Carnot}
\begin{document}
\maketitle
\tableofcontents

\part{Algèbre}
\chapter{Structure de groupe}
	\section{Présentation}
		\subsection{Exemple préliminaire}
		\subsection{Définition générale}
		\subsection{Exemples usuels}
			\subsubsection{Ensembles de nombres}
			\subsubsection{Ensemble des bijections}
			\subsubsection{Ensemble des parties}
		\subsection{Compléments}
			\subsubsection{Unicité}
			\subsubsection{Formules concernant le symétrique}
			\subsubsection{Régularité de tout élément}
			\subsubsection{Plus généralement}
		\subsection{Notations}
			\subsubsection{En notation multiplicative}
			\subsubsection{En notation additive}
			\subsubsection{Propriétés}
		\subsection{Autres remarques}
	\section{Sous-groupes}
		\subsection{Définition}
		\subsection{Caractérisations}
		\subsection{Exemples usuels}
			\subsubsection{Exemple général}
			\subsubsection{Exemples particuliers}
			\subsubsection{Sous-groupes de $(\mathbb{Z},+)$}
		\subsection{Propriétés}
			\subsubsection{Intersection}
			\subsubsection{Faux pour la réunion $\cup$}
	\section{Morphismes de groupes}
		\subsection{Définition}
		\subsection{Exemples usuels}
		\subsection{Propriétés}
		
\chapter{Structure d'anneau et de corps}
	\section{Structure d'anneau}
		\subsection{Présentation}
			\subsubsection{Exemple préliminaire}
			\subsubsection{Définition générale}
			\subsubsection{Notations}
			\subsubsection{Intégrité}
			\subsubsection{Exemples usuels}
		\subsection{Propriétés}
			\subsubsection{Élément absorbant}
			\subsubsection{Ensemble des inversibles}
			\subsubsection{"Opposé" d'un produit}
			\subsubsection{Loi "soustraction"}
			\subsubsection{Formule du binôme de Newton}
			\subsubsection{Formule de factorisation}
		\subsection{Sous-anneau}
			\subsubsection{Caractérisation}
			\subsubsection{Exemple usuel : sous-anneau des décimaux}
	\section{Structure de corps}
		\subsection{Définition}
		\subsection{Exemples usuels}
		\subsection{Propriétés}
			\subsubsection{Intégrité}
			\subsubsection{Commutativité}
		\subsection{Sous-corps}
			
\chapter{Corps des nombres réels}
	\section{Généralités}
	\section{Borne supérieure ou inférieure d'une partie de $\mathbb{R}$}
		\subsection{Définition}
		\subsection{Existence-unicité}
			\subsubsection{Existence}
			\subsubsection{Unicité}
		\subsection{Mise en garde}
		\subsection{Caractérisation}
	\section{Valeurs approchées d'un réel à $\alpha$ près (où $\alpha \in \mathbb{Q}^{\ast+}$)}
		\subsection{Résultat et définition}
		\subsection{Cas où $\alpha = 1$}
		\subsection{Cas où $\alpha = \frac{1}{10^{n}} (n \in \mathbb{N})$}
			\subsubsection{Énoncé}
			\subsubsection{Convergence}
	\section{Densité}
		\subsection{Définitions}
			\subsubsection{Intervalle}
			\subsubsection{Densité}
		\subsection{Caractérisation}
			\subsubsection{Générale}
			\subsubsection{Plus précisément}
			\subsubsection{Remarques}
			\subsubsection{Exemple usuel}
		\subsection{Compléments}
			\subsubsection{Concernant $\mathbb{Q}$}
			\subsubsection{Concernant$\bar{\mathbb{Q}}$ (complément de $\mathbb{Q}$ dans $\mathbb{R}$)}
			
\chapter{Corps des nombres complexes}
	\section{Conjugaison}
		\subsection{Définition}
		Soit $z = a+ib \in \mathbb{C}$. Son conjugué est : $\bar{z} = a - ib$
		\subsection{Propriétés}
			\subsubsection{Formules}
			$\forall z \in \mathbb{C}, \ z+ \bar{z} = 2\mbox{Re}(z), \ z - \bar{z} = 2i\mbox{Im}(z)$
		
			\subsubsection{Caractérisation}
				\begin{equation} \begin{split}
						z \in \mathbb{R} &\Leftrightarrow z = \bar{z} \\
						z \in i\mathbb{R} &\Leftrightarrow z = -\bar{z}
					\end{split}
				\end{equation} 
			\subsubsection{Pratique}
				Quand un nombre complexe est écrit au dénominateur, on le multiplie par son conjugué.
	\section{Module}
		\subsection{Définition}
			$\forall z \in \mathbb{C}$, le module de z est $\left| z\right| = \sqrt{z \bar{z}}$
			\subsubsection{Pratique}
			Pour $z=a+ib$, on a $\left| z\right| = \sqrt{a^2 + b^2}$
			\subsubsection{Lien avec la valeur absolue}
			Le module dans $\mathbb{C}$ prolonge la valeur absolue dans $\mathbb{R}$.
		\subsection{Propriétés}
			\subsubsection{Diverses}
			\[\begin{split}
				\forall z \in \mathbb{C},\left| z\right| \geqslant 0 &\mbox{ et } \left| z\right| = 0 \Leftrightarrow z= 0 \\
				\left| z \right| &= \left| \bar{z} \right| \\
				\forall z,z' \in \mathbb{C},\left| zz' \right|& = \left| z \right| \left| z' \right| 
			\end{split} \]
			\subsubsection{(Double) inégalité triangulaire}
			\[\forall z,z' \in \mathbb{C},\ \left|\left|z\right|-\left|z'\right|\right|\leqslant\left|z+z'\right| \leqslant \left|z\right| + \left|z'\right|\]
		\subsection{Nombres complexes de module 1}
			\subsubsection{Description}
			$U = \left\{z \in \mathbb{C}, \ \left|z\right| = 1\right\}$. Les complexes de module 1 s'écrivent $z = \cos(\theta)+i\sin(\theta)$, ce qu'on note : $ e^{i\theta}$ où $\theta \in \mathbb{R} \cdot\setminus 2\pi \mathbb{Z}$
			\subsubsection{Remarque sur l'écriture $e^{i\theta}$}
			$\forall \theta, \ e^{-i\theta} = \bar{e^{i\theta}} = \frac{1}{e^{i\theta}} $
			\subsubsection{Produit}
			$\forall \theta, \theta', \ e^{i\theta} e^{i\theta'} = e^{i(\theta+\theta')}$
			\subsubsection{Formule de Moivre}
			\subsubsection{Formules à savoir}
	\section{Forme trigonométrique}
		\subsection{Définition}
			\subsubsection{Résultat préliminaire}
			\subsubsection{Conséquence}
		\subsection{Premiers exemples}
			\subsubsection{Divers}
			\subsubsection{Caractérisations}
		\subsection{Relations entre forme algébrique et trigonométrique}
			\subsubsection{Sens direct}
			\subsubsection{Sens réciproque}
		\subsection{Formules diverses}
		\subsection{Interprétation géométrique}
	\section{Équation $z^{n}=a$ (où $n\in \mathbb{N}^{\ast},a\in \mathbb{C}^{\ast}$)}
		\subsection{Résolution}
			\subsubsection{Solutions confondues}
			\subsubsection{Plus précisément}
			\subsubsection{Ces $n$ solutions sont bien distinctes}
			\subsubsection{Conclusion}
			\subsubsection{Géométriquememnt}
		\subsection{1\up{er} cas parrticulier : racines n\up{èmes} de l'unité}
			\subsubsection{Définitionn}
			\subsubsection{Description}
			\subsubsection{Propriétés}
			\subsubsection{Interprétation géométrique}
			\subsubsection{Cas $n=3$}
		\subsection{Cas particulier des racines carrées d'un complexe}
			\subsubsection{Énoncé}
			\subsubsection{Obtention pratique}
			\subsubsection{Équation de d°2 dans $\mathbb{C}$}
	\section{Traduction complexe de transformations géométriques}
		\subsection{Symétries}
			\subsubsection{Par rapport à $O\overrightarrow{i}$}
			\subsubsection{Centrale par rapport à $O$}
			\subsubsection{Par rapport à $O\overrightarrow{j}$}
		\subsection{Translations}
		\subsection{Homothéties}
			\subsubsection{Définition}
			\subsubsection{Traduction complexe}
		\subsection{Rotations}
			\subsubsection{Définition}
			\subsubsection{Traduction complexe}
		\subsection{Similitudes directes}
			\subsubsection{Définition}
			\subsubsection{Traduction complexe}
	\section{Exponentielle complexe}
		\subsection{Définition}
		\subsection{Propriétés}
			\subsubsection{Module - argument}
			\subsubsection{Formule fondamentale}
			\subsubsection{Résolution de l'équation $e^{z} = a$}
			\subsubsection{Égalité}
			
\chapter{Anneau $\mathbb{K}[X]$ des polynômes à une indéterminée à coefficients dans un corps $\mathbb{K}$}
	\section{Présentation}
		\subsection{Définitions}
		\subsection{Opérations sur les polynômes}
			\subsubsection{Somme}
			\subsubsection{Multiplication par un élément de $\mathbb{K}$}
			\subsubsection{Multiplication}
		\subsection{Propriétés}
			\subsubsection{Pour la multiplication}
			\subsubsection{Pour la somme}
		\subsection{Structures}
			\subsubsection{Neutres}
			\subsubsection{Intégrité}
			\subsubsection{Inversibles}
		\subsection{Composée}
			\subsubsection{Définition}
			\subsubsection{Degré}
	\section{Division euclidienne dans $\mathbb{K}[X]$}
		\subsection{Énoncé}
			\subsubsection{Unicité}
			\subsubsection{Existence}
		\subsection{Exemples}
		\subsection{Divisibilité}
			\subsubsection{Définition}
			\subsubsection{Divisibilité par $X-a$}
	\section{PGCD,PPCM dans $\mathbb{K}[X]$}
		\subsection{Définition pour PGCD}
			\subsubsection{Remarques préliminaires}
			\subsubsection{Algorithmique}
			\subsubsection{Définition générale}
		\subsection{Propriétés}
			\subsubsection{Relation de Bézout}
			\subsubsection{Formule de factorisation}
		\subsection{Polynômes premiers entre eux}
			\subsubsection{Définition}
			\subsubsection{Caractérisation : théorème de Bézout}
			\subsubsection{Propriétés}
		\subsection{PPCM dans $\mathbb{K}[X]$}
			\subsubsection{Données du problème}
			\subsubsection{Résolution}
			\subsubsection{Conclusion}
	\section{Zéros (ou racines) d'un polynôme}
		\subsection{Définitions}
		\subsection{Relation entre les racines et le degré d'un polynôme}
			\subsubsection{Résultat}
			\subsubsection{Conséquence}
		\subsection{Polynôme dérivé}
			\subsubsection{Définition}
			\subsubsection{Formule de Taylor}
		\subsection{Caractérisation d'un zéro d'ordre $n$}
	\section{Polynômes irréductibles}
		\subsection{Présentation}
			\subsubsection{Définitions}
			\subsubsection{Remarques}
			\subsubsection{Attention au corps considéré}
		\subsection{Décomposition générale}
			\subsubsection{Résultat préliminaire}
			\subsubsection{Existence}
			\subsubsection{Unicité}
			\subsubsection{Première application}
		\subsection{Dans $\mathbb{C}[X]$}
			\subsubsection{Résultat admis}
			\subsubsection{Précision}
			\subsubsection{Décomposition en facteurs irréductibles unitaires dans $\mathbb{C}[X]$}
			\subsubsection{Conséquences}
			\subsubsection{Pratique}
			\subsubsection{Exemple usuel}
		\subsection{Dans $\mathbb{R}[X]$}
			\subsubsection{Polynôme conjugué}
			\subsubsection{Conséquences}
			\subsubsection{Ainsi :}
		\subsection{Pratique de la décomposition en facteurs irréductibles dans $\mathbb{R}[X]$}
			\subsubsection{Utiliser les racines}
			\subsubsection{Utiliser des identités remarquables}
			\subsubsection{Utiliser la méthode du bicarré}
	\section{Relations coefficients-racines}
		\subsection{Données du problème}
			\subsubsection{Rappel}
			\subsubsection{But : généraliser}
			\subsubsection{Définitions et notations}
			\subsubsection{But}
		\subsection{Résolution}
		\subsection{Appplications}
		
\chapter{Fractions rationnelles}
	\section{Présentation}
		\subsection{Définition}
		\subsection{Opérations}
			\subsubsection{Somme}
			\subsubsection{Produit}
			\subsubsection{Structure}
		\subsection{Forme irréductible}
	\section{Décomposition en éléments simples de $F = \frac{A}{B}$ (irréductible)}
		\subsection{Première étape : partie entière}
			\subsubsection{Énoncé}
			\subsubsection{Démonstration par Analyse-Synthèse}
		\subsection{Deuxième étape : décomposition de $\frac{R}{B}$}
			\subsubsection{Énoncé}
			\subsubsection{Démonstration}
		\subsection{Troisième étape : généralisation}
		\subsection{Conséquence}
		\subsection{Quatrième étape : décomposition de $\frac{R}{P^{\alpha}}$}
			\subsubsection{Résultat général}
			\subsubsection{Démonstration}
		\subsection{Conclusion}
	\section{Décomposition dans $\mathbb{C}(X)$}
		\subsection{Forme a priori}
		\subsection{Détermination pratique des $\lambda,\mu$}
		\subsection{Exemple usuel particulier}
		\subsection{Exemple usuel général}
	\section{Dans $\mathbb{R}(X)$}
		\subsection{Forme a priori}
		\subsection{Détermination pratique des $\lambda,\alpha,\beta$}
		\subsection{Exemple usuel}
	\section{Application principale : calculs de primitive de fonctions rationelles}
		\subsection{Définition}
		\subsection{Méthode pour primitiver $f(x)=\frac{P(x)}{Q(x)} \in \mathbb{R}(x)$}
		
\chapter{Groupe symétrique}
	\section{Présentation}
		\subsection{Définitions}
			\subsubsection{Permutation}
			\subsubsection{Groupe symétrique}
			\subsubsection{Cardinal}
		\subsection{Exemples}
			\subsubsection{Généraux}
			\subsubsection{Particulier}
	\section{Éléments générateurs}
		\subsection{Transpositions}
			\subsubsection{Énoncé}
			\subsubsection{Exemples}
		\subsection{Cycles à supports disjoints}
			\subsubsection{Résultat admis}
			\subsubsection{Exemple}
			\subsubsection{Pratique}
	\section{Signature d'une permutation}
		\subsection{Inversions}
		\subsection{Définitions}
		\subsection{Cas d'une transposition}
		\subsection{Cas d'un cycle}
		\subsection{Morphisme signature}
			





\end{document}