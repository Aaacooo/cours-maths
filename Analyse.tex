\documentclass[12pt,a4paper,french]{book}
\usepackage{graphicx} % Required for inserting images
\usepackage{multirow}
\usepackage[frenchb]{babel}
\usepackage{fancybox,framed}
\usepackage{amssymb}
\usepackage{amsmath}
\usepackage{array}
\title{Cours}
\author{C. LACOUTURE}
\date{Année scolaire 2024-2025, MPSI2, Lycée Carnot}
\begin{document}
\maketitle
\tableofcontents
\part{Analyse}
\chapter{Suites réelles (ou complexes)}
	\section{Majoration, minoration}
		\subsection{Définitions}
		\subsection{Caractérisation}
	\section{Limite}
		\subsection{Définition préliminaire : voisinage d'un point}
			\subsubsection{Voisinage d'un point $a \in \mathbb{R}$}
			\subsubsection{Voisinage de $+\infty$}
			\subsubsection{Voisinage de $-\infty$}
		\subsection{Définitions}
			\subsubsection{Générale}
			\subsubsection{Ainsi :}
			\subsubsection{Convergence, divergence}
		\subsection{Théorèmes relatifs à la valeur absolue}
		\subsection{Théorèmes relatifs à la relation d'ordre $\leqslant$}
		\subsection{Théorèmes opératoires}
			\subsubsection{Relatifs à l'addition}
			\subsubsection{Relatifs à la multiplication}
			\subsubsection{Relatifs au quotient}
	\section{Suites usuelles}
		\subsection{Suites arithmétiques}
			\subsubsection{Définition, caractérisation}
			\subsubsection{Limite}
		\subsection{Suites géométriques}
			\subsubsection{Définition, caractérisation}
			\subsubsection{Démonstrations}
			\subsubsection{Limite}
		\subsection{Suites arithmético-géométriques}
			\subsubsection{Définition}
			\subsubsection{Expression}
	\section{Monotonie}
		\subsection{Définitions}
			\subsubsection{Générales}
			\subsubsection{Cas particulier}
		\subsection{Théorème fondamental}
		\subsection{À savoir}
	\section{Deux notions complémentaires}
		\subsection{Suites adjacentes}
			\subsubsection{Définition}
			\subsubsection{Propriété}
			\subsubsection{Exemple usuel : approximation d'un réel à $10^{-n}$ près}
		\subsection{Suites extraites}
			\subsubsection{Définition}
			\subsubsection{(Contre) execos(n)mp, le}
			\subsubsection{Suite extraite et limite}
			\subsubsection{Théorème de Bolzano-Weierstrass}
		\subsection{Applications}
			\subsubsection{Divergence de $(\cos(n))$,$(\sin(n))$}
			\subsubsection{Divergence de la série harmonique}
			\subsubsection{Séries alternées}
			\subsubsection{Série exponentielle}
	\section{Théorème de la moyenne de Cesaro}
		\subsection{Énoncé}
		\subsection{Démonstration}
	\section{Suites récurrentes}
		\subsection{Définition}
		\subsection{Remarque}
		\subsection{Méthode générale}
		
\chapter{Fonctions numériques (ou à valeurs complexes) d'une variable réelle, limite, continuité}
	\section{Limite}
		\subsection{Définition générale}
		\subsection{Traductions}
		\subsection{Théorèmes}
			\subsubsection{Théorème de passages aux limites finies dans une inégalité}
			\subsubsection{Théorème des limites finies par encadrement}
		\subsection{Théorèmes de la limite monotone}
	\section{Continuité}
		\subsection{En un point}
			\subsubsection{Définition}
			\subsubsection{Continuité à gauche, à droite seulement}
			\subsubsection{Prolongement par continuité}
			\subsubsection{Caractérisation séquentielle}
		\subsection{Sur un intervalle}
			\subsubsection{Définition}
			\subsubsection{Théorèmes opératoires}
			\subsubsection{Composée}
			\subsubsection{Pratique}
	\section{Continuité et intervalle}
		\subsection{Résultat général}
			\subsubsection{Théorème des valeurs intermédiaires}
			\subsubsection{Démonstration}
		\subsection{Pour un segment}
			\subsubsection{Énoncé}
			\subsubsection{Démonstration}
			\subsubsection{Remarques}
			\subsubsection{Application}
			\subsubsection{Contre-exemple}
		\subsection{Pour une fonction strictement monotone}
			\subsubsection{Remarque générale}
			\subsubsection{Forme de l'intervalle image}
			\subsubsection{Stricte monotonie de $f^{-1}$}
			\subsubsection{Continuité de $f^{-1}$}
			\subsubsection{Remarque graphique}
			\subsubsection{Remarque éventuelle d'imparité}
			\subsubsection{Résumé}
			\subsubsection{Condition suffisante}
			
\chapter{Relations de comparaison}
	\section{Sur les suites}
		\subsection{Définitions}
			\subsubsection{Domination}
			\subsubsection{Négligeabilité}
			\subsubsection{Équivalence}
			\subsubsection{Traduction pratique}
		\subsection{Propriétés}
			\subsubsection{Relation d'équivalence}
			\subsubsection{Équivalence et limite}
			\subsubsection{Équivalence et signe}
			\subsubsection{Équivalence et opérations}
			\subsubsection{Équivalence et somme}
			\subsubsection{Équivalence et négligeabilité}
			\subsubsection{Équivalence par encadrement}
			\subsubsection{Mise en garde}
		\subsection{Relations de négligeabilité usuelles}
	\section{Sur les fonctions d'une variable réelle}
		\subsection{Définitions}
			\subsubsection{Domination}
			\subsubsection{Négligeabilité}
			\subsubsection{Équivalence}
			\subsubsection{En pratique}
		\subsection{Propriétés}
		\subsection{Exemples usuels de négligeabilité}
		\subsection{Exemples usuels d'équivalence}
			\subsubsection{Fonctions polynômes}
			\subsubsection{Équivalents usuels avec les limites usuels en 0}
			\subsubsection{Problème supplémentaire}

\chapter{Dérivabilité d'une fonction numérique (ou à valeurs complexes) définie sur un intervalle de $\mathbb{R}$}
	\section{Présentation}
		\subsection{En un point}
			\subsubsection{Définition}
			\subsubsection{Dérivabilité à gauche, à droite}
			\subsubsection{Lien avec la continuité}
			\subsubsection{Interprétation graphique}
		\subsection{Sur un intervalle}
			\subsubsection{Définition}
			\subsubsection{Précision}
		\subsection{Théorèmes}
			\subsubsection{Combinaison linéaire}
			\subsubsection{Produit}
			\subsubsection{Quotient}
			\subsubsection{Composée}
			\subsubsection{Réciproque}
	\section{Dérivées d'ordres supérieurs}
		\subsection{Définitions}
			\subsubsection{Générale}
			\subsubsection{Fonctions de classe $\mathcal{C}^{k}$}
			\subsubsection{Contre-exemple de fonction dérivable, mais pas $\mathcal{C}^{1}$ sur $I$}
			\subsubsection{Propriétés simples}
			\subsubsection{Exemples}
		\subsection{Théorèmes}
			\subsubsection{Combinaison linéaire}
			\subsubsection{Formule de Leibniz}
			\subsubsection{Quotient}
			\subsubsection{Composée}
			\subsubsection{Réciproque}
	\section{Théorèmes de Rolle et application}
		\subsection{Présentation}
			\subsubsection{Énoncé}
			\subsubsection{Graphiquement}
			\subsubsection{Démonstration}
			\subsubsection{Application immédiate}
		\subsection{Autres applications : le théorème des accroissements finis (TAF)}
			\subsubsection{Énoncé}
			\subsubsection{Graphiquement}
			\subsubsection{Remarque}
			\subsubsection{Démonstration}
		\subsection{"Applications d'applications"}
			\subsubsection{Inégalité des accroissements finis}
			\subsubsection{Théorème de limite de la dérivée}
			\subsubsection{Fonctions lipschitziennes}
	\section{Monotonie et extrema}
		\subsection{Monotonie}
			\subsubsection{Définitions}
			\subsubsection{Caractérisation}
		\subsection{Extrema}
			\subsubsection{Définitions}
			\subsubsection{Condition nécessaire d'existence}
			\subsubsection{Remarque}
			\subsubsection{Condition suffisante d'existence}
\chapter{Fonctions trigonométriques}
	\section{Rappel et compléments sur les fonctions sinus et cosinus}
		\subsection{Valeurs usuelles}
		\subsection{Formules élémentaires}
		\subsection{Parité, périodicité}
		\subsection{Égalités}
			\subsubsection{De cosinus}
			\subsubsection{De sinus}
		\subsection{cosinus et sinus de sommes}
			\subsubsection{Formules}
			\subsubsection{Conséquences}
			\subsubsection{Écriture}
		\subsection{Produit de cosinus, sinus}
		\subsection{Sommes de cosinus, sinus}
		\subsection{Compléments}
			\subsubsection{Limites usuelles}
			\subsubsection{Inégalités usuelles}
			\subsubsection{Transformation de $a\cos + b\sin$}
	\section{Fonction tangente}
		\subsection{Étude de la fonction}
			\subsubsection{Définition et domaine de définition}
			\subsubsection{Remarques de parités, périodicité}
			\subsubsection{Continuité}
			\subsubsection{Dérivabilité}
			\subsubsection{Variations}
			\subsubsection{Graphe}
			\subsubsection{Limite usuelle}
		\subsection{Compléments}
			\subsubsection{Valeurs usuelles}
			\subsubsection{Égalité}
			\subsubsection{Formules immédiates}
			\subsubsection{Tangente d'une somme}
			\subsubsection{Expressions de cosinus et sinus à l'aide de la tangente de l'angle moitié}
			\subsubsection{Interprétation graphique}
\chapter{Fonctions circulaires réciproques}subsection{}
	\section{Fonction arcsinus}
		\subsection{Définition}
		\subsection{Variations et graphe}
		\subsection{Formules}
		\subsection{Dérivation}
	\section{Fonction arccosinus}
		\subsection{Définition}
		\subsection{Variations et graphe}
		\subsection{Formules}
		\subsection{Dérivation}
		\subsection{Égalités supplémentaires}
	\section{Fonction arctan}
		\subsection{Définition}
		\subsection{Variations et graphe}
		\subsection{Formules}
		\subsection{Dérivation}
		\subsection{Égalité supplémentaire}

\chapter{Fonctions logarithmes, exponentielles, puissances réelles}
	\section{Fonction logarithme}
		\subsection{Définition}
		\subsection{Formules}
		\subsection{Étude de la fonction ln}
			\subsubsection{Ensemble de définition}
			\subsubsection{Continuité, dériivabilité}
			\subsubsection{Variations}
			\subsubsection{Branches infinies}
			\subsubsection{Graphe}
			\subsubsection{Compléments}
	\section{Fonction exponentielle}
		\subsection{Définition}
		\subsection{Formule}
		\subsection{Graphiquement}
		\subsection{Dérivée}
	\section{Fonctions puissances réelles}
		\subsection{Définition}
			\subsubsection{Prolongement}
			\subsubsection{Définition de la puissance rationelle}
			
			\subsubsection{Domaine de définition de la puissance rationelle}
			
		\subsection{Formules}
		
		\subsection{Étude de $x\mapsto x^{\alpha} = f_{\alpha}$}
			\subsubsection{Domaine de définition}
			\subsubsection{Continuité}
			\subsubsection{Dérivabilité}
			\subsubsection{Variations}
			\subsubsection{Branches infinies}
			\subsubsection{Graphes}
	\section{Relations de comparaisons  (ou croissances comparées)}
		\section{Résultats généraux}
		\section{Autres situations}
\chapter{Fonctions hyperboliques}
	\section{Présentation}
		\subsection{Définitions}
			\subsubsection{Fonction sinus hyperbolique : sh}
			\subsubsection{Fonction cosinus hyperbolique : ch}
			\subsubsection{Fonction tangente hyperbolique : th}
		\subsection{Formules de base}
	\section{Étude de la fonction sh}
		\subsection{Domaine de définition}
		\subsection{Remarque de parité}
		\subsection{Continuité}
		\subsection{Dérivabilité}
		\subsection{Variations}
		\subsection{Branches infinies}
		\subsection{Graphe}
		\subsection{Limite usuelle}
	\section{Étude de la fonction ch}
		\subsection{Domaine de définition}
		\subsection{Remarque de parité}
		\subsection{Continuité}
		\subsection{Dérivabilité}
		\subsection{Variations}
		\subsection{Branches infinies}
		\subsection{Graphe}
		\subsection{Limite usuelle}
	\section{Étude de la fonction th}
		\subsection{Domaine de définition}
		\subsection{Remarque de parité}
		\subsection{Continuité}
		\subsection{Dérivabilité}
		\subsection{Variations}
		\subsection{Branches infinies}
		\subsection{Graphe}
		\subsection{Limite usuelle}
	\section{Petit formulaire hyperbolique}
\chapter{Fonctions convexes}
	\section{Présentation}
		\subsection{Définition}
			\subsubsection{Générale}
			\subsubsection{Remarques}
			\subsubsection{Généralisation avec l'inégalité de Jensen}
	\section{Caractérisations géométriques}
		\subsection{Croissances des pentes dont on fixe une extrémité}
		\subsection{Convexité de la partie du plan située au dessus de la courbe}
			\subsubsection{Définition géométrique de la convexité}
			\subsubsection{Énoncé}
			\subsubsection{Démonstration}
	\section{Caractérisations analytiques à l'aide de dérivées}
		\subsection{Rapport entre convexité et dérivation}
			\subsubsection{Sens direct}
			\subsubsection{Sens réciproque}
			\subsubsection{Conséquence}
	\section{Premières applications}
		\subsection{Inégalité usuelle}
		\subsection{Comparaison des moyennes arithmétiques et géométriques}
		
\chapter{Intégration d'une fonction numérique (ou à valeurs complexes) sur un segment}
	\section{Fonctions uniformément continues}
		\subsection{Définition}
		\subsection{Caractérisation séquentielle de l'UC}
		\subsection{Premières propriétés}
			\subsubsection{Combinaison linéaire}
			\subsubsection{Implications}
		\subsection{Théorème de Heine}
	\section{Définitions}
		\subsection{Subdivision}
			\subsubsection{Définition générale}
			\subsubsection{Subdivision à pas constant}
		\subsection{Fonctions en escalier}
			\subsubsection{Définition}
			\subsubsection{Exemple}
		\subsection{Fonction continue par morceaux}
			\subsubsection{Définition}
			\subsubsection{Graphiquement}
	\section{Premières propriétés}
	\section{Intégration des fonctions en escalier}
		\subsection{Définition}
		\subsection{Propriétés}
	\section{Définition générale de l'intégrale d'une fonction continue par morceaux}
		\subsection{Approximation d'une fonction continue par morceaux par des fonctions en escaliers}
		\subsection{Définition de l'intégrale de $f \in \mathcal{C}_m([a,b])$}
		\subsection{Propriétés}
			\subsubsection{Linéarité}
			\subsubsection{Positivité, croissance}
			\subsubsection{Additivité par rapport à l'intervalle d'intégration}
	\section{Intégration sur un intervalle quelconque}
		\subsection{Définition}
		\subsection{Propriétés}
			\subsubsection{Linéarité}
			\subsubsection{Additivité par rapport à l'intervalle d'intégration}
	\section{Compléments pour des fonctions continues}
		\subsection{Théorème fondamental}
		\subsection{Formule de la moyenne}
		\subsection{Théorème de positivité amélioré}
		\subsection{Sommes de Riemann}
		\subsection{IPP, changement de variables}
			\subsubsection{Formule d'Intégration Par Parties}
			\subsubsection{Formule de changement de variable}
		\subsection{Autres formules de Taylor}
			\subsubsection{Formule de Taylor avec reste intégral}
			\subsubsection{Inégalité de Taylor-Lagrange}

\chapter{Développements limités}
	\section{Formule de Taylor-Young}
		\subsection{Énoncé}
		\subsection{Démonstration}
			\subsubsection{But}
			\subsubsection{Résolution}
		\subsection{Applications}
	\section{Définitions}
		\subsection{En 0}
		\subsection{En $x_0$}
		\subsection{En $\infty$}
		\subsection{Dans la suite}
	\section{Premières propriétés}
		\subsection{DL et convergence, dérivabilité}
			\subsubsection{Convergence}
			\subsubsection{Dérivabilité}
			\subsubsection{Contre-exemple}
		\subsection{Ordre de DL}
		\subsection{Unicité}
		\subsection{Conséquence : remarque de parité}
	\section{DL usuels}
		\subsection{Obtention}
		\subsection{Énumération}
		\subsection{Remarques}
	\section{Opérations sur les DL}
		\subsection{Combinaison linéaire}
		\subsection{Produit}
		\subsection{Intégration}
			\subsubsection{Énoncé}
			\subsubsection{Démonstration}
			\subsubsection{Conséquences}
		\subsection{Composée de DL}
		\subsection{Quotient}
		
\chapter{Primitives et équations différentielles}
	\section{Primitives}
		\subsection{Rappels}
			\subsubsection{Définition}
			\subsubsection{Propriété}
			\subsubsection{Existence}
			\subsubsection{Primitives usuelles}
		\subsection{Formules d'Intégration par parties (IPP)}
			\subsubsection{Énoncé}
			\subsubsection{Autre primitive usuelle}
		\subsection{Calcul de $\int e^{\alpha x}P(x)dx$}
			\subsubsection{1\up{ère} méthode}
			\subsubsection{2\up{ème} méthode}
		\subsection{Calcul de $\int e^{\alpha x}\cos(ax)dx$, $\int e^{\alpha x}\sin(ax)dx$}
			\subsubsection{1\up{ère} méthode}
			\subsubsection{2\up{ème} méthode}
		\subsection{Calcul de $\int \frac{1}{(x-a_1)(x-a_2)\dots(x-a_n)}dx$}
			\subsubsection{1\up{ère} méthode}
			\subsubsection{2\up{ème} méthode}
		\subsection{Calcul de $\int \frac{1}{ax^{2}+bx+c}dx$}
			\subsubsection{Méthode}
			\subsubsection{Exemple}
	\section{Présentation des équations différentielles}
	\section{Équations différentielles d'ordre 1}
		\subsection{Définition}
		\subsection{Résolution de l'équation différentielle homogène}
		\subsection{Résolution de l'équation différentielle complète}
			\subsubsection{À l'aide d'une solution particulière $y_0$}
			\subsubsection{Avec la méthode de "variation de la constante"}
			\subsubsection{Coïncidence des 2 méthodes}
			\subsubsection{Problème d'existence-unicité}
	\section{Équations différentielles linéaires d'ordre 2 à coefficients constants}
		\subsection{Définition}
		\subsection{Résolution de l'équation différentielle homogène}
			\subsubsection{Résultat admis en 1\up{ère} année}
			\subsubsection{Cas où $\Delta = b^{2}-4ac > 0$}
			\subsubsection{Cas où $\Delta = b^{2}-4ac = 0$}
			\subsubsection{Cas où $\Delta = b^{2}-4ac < 0$}
			\subsubsection{Remarques dans le cas complexe}
		\subsection{Résolution de l'équation différentielle complète}
			\subsubsection{Principe}
			\subsubsection{Dans le cas où le second membre $d(x)$ est de la forme  $e^{\gamma x}P(x)dx$}
			\subsubsection{Dans le cas où le second membre $d(x)$ est une somme d'expressions de la forme}
		\subsection{Existence-unicité}
			\subsubsection{Plus précisément}
			\subsubsection{Démonstration par Analyse-Synthèse}
			
\chapter{Séries numériques}
\chapter{Familles sommables}
	\section{Dans le cas d'une famille de réels $\geqslant 0$}
		\subsection{Définition}
		\subsection{Remarques}
		\subsection{Propriétés}
		\subsection{À savoir en pratique}
	\section{Pour une famille de réels quelconques}
		\subsection{Définition}
		\subsection{Caractérisation}
		\subsection{À savoir en pratique}
		\subsection{Contre-exemple}
	\section{Pour une famille de termes complexes}
		\subsection{Définition}
		\subsection{Caractérisation}
		\subsection{À savoir}
		
\chapter{Fonctions de deux variables}



	
	
	
	
	
	
	
	




\end{document}