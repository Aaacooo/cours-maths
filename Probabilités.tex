\documentclass[12pt,a4paper,french]{book}
\usepackage{graphicx} % Required for inserting images
\usepackage{multirow}
\usepackage[frenchb]{babel}
\usepackage{fancybox,framed}
\usepackage{amssymb}
\usepackage{amsmath}
\usepackage{array}
\title{Cours}
\author{C. LACOUTURE}
\date{Année scolaire 2024-2025, MPSI2, Lycée Carnot}
\begin{document}
\maketitle
\tableofcontents
\part{Probabilités}

\chapter{Probabilités sur un univers fini}
	\section{Présentation}
		\subsection{Définitions}
			\subsubsection{Générale}
			\subsubsection{Définitions complémentaires}
		\subsection{Propriétés simples}
			\subsubsection{Complémentaire}
			\subsubsection{Inclusion}
			\subsubsection{Réunion quelconque}
			\subsubsection{Événements incompatibles deux à deux}
		\subsection{Pratique}
			\subsubsection{De manière générale}
			\subsubsection{Probabilité uniforme sur $\Omega$}
	\section{Probabilités conditionnelles}
		\subsection{Définitions}
			\subsubsection{Générale}
			\subsubsection{Vérification}
		\subsection{Formules}
			\subsubsection{Formule des probabilités composées}
			\subsubsection{Formule des probabilités totales}
			\subsubsection{Formules de Bayes}
	\section{Événements indépendants}
		\subsection{Définition}
		\subsection{Propriétés}
		\subsection{Mutuelle indépendance}

\chapter{Variables aléatoires sur un espace probabilisé fini}
	\section{Présentation}
		\subsection{Définitions}
			\subsubsection{Générale}
			\subsubsection{Événement}
			\subsubsection{Loi de probabilité $p_X$}
		\subsection{Pratique}
			\subsubsection{Événements élémentaires}
			\subsubsection{Propriété}
			\subsubsection{Conséquences}
		\subsection{Composée}
	\section{Lois classiques}
		\subsection{Loi uniforme}
			\subsubsection{Définition}
			\subsubsection{Conséquence}
		\subsection{Loi de Bernoulli}
			\subsubsection{Définition}
			\subsubsection{Exemple}
		\subsection{Loi binomiale}
			\subsubsection{Définition}
			\subsubsection{Exemple}
	\section{Couples de variables aléatoires}
		\subsection{Définition}
		\subsection{Pratique}
		\subsection{Variables aléatoires indépendantes}
			\subsubsection{Définition}
			\subsubsection{Conséquences sur des événements quelconques}
		\subsection{Mutuelle indépendance}
			\subsubsection{Définition}
			\subsubsection{Caractérisation}
		\subsection{Composée}
			\subsubsection{Pour deux variables aléatoires}
			\subsubsection{Extension}
	\section{Espérance d'une variable aléatoire}
		\subsection{Définition}
			\subsubsection{Initiale}
			\subsubsection{Autre formulation}
			\subsubsection{Égalité des deux formules}
		\subsection{Propriétés}
			\subsubsection{"Linéarité"}
			\subsubsection{Positivité}
			\subsubsection{Croissance}
		\subsection{Espérances des lois classiques}
			\subsubsection{Variable aléatoire constante}
			\subsubsection{Loi de Bernoulli}
			\subsubsection{Loi binomiale}
		\subsection{Autres formules}
			\subsubsection{Formule du théorème de transfert}
			\subsubsection{Inégalité de Markov}
			\subsubsection{Condition nécessaire d'indépendance}
	\section{Variance}
		\subsection{Définitions}
			\subsubsection{Initiale}
			\subsubsection{Autre formulation}
			\subsubsection{Calculs}
			\subsubsection{Écart type}
		\subsection{Propriétés}
			\subsubsection{Formule}
			\subsubsection{Inégalité de Bienaymé-Tchebychev}
	\section{Covariance}
		\subsection{Définition}
		\subsection{Propriétés}
			\subsubsection{Condition nécessaire}
			\subsubsection{Symétrie}
			\subsubsection{Bilinéarité}
			\subsubsection{Lien avec la variance}
			\subsubsection{Conséquence}
		\subsection{Variances usuelles}
			\subsubsection{Variable aléatoire constante}
			\subsubsection{Loi de Bernoulli}
			\subsubsection{Loi binomiale}
	
	
\end{document}