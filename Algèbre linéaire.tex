\documentclass[12pt,a4paper,french]{book}
\usepackage{graphicx} % Required for inserting images
\usepackage{multirow}
\usepackage[frenchb]{babel}
\usepackage{fancybox,framed}
\usepackage{amssymb}
\usepackage{amsmath}
\usepackage{array}
\title{Cours}
\author{C. LACOUTURE}
\date{Année scolaire 2024-2025, MPSI2, Lycée Carnot}
\begin{document}
\maketitle
\tableofcontents

\part{Algèbre linéaire}

\chapter{Structure d'espace vectoriel sur un corps $\mathbb{K}$}
	\section{Présentation}
		\subsection{Exemple prélimminaire}
		\subsection{Définition générale}
		\subsection{Propriétés élémentaires}
		\subsection{Exemples}
			\subsubsection{Usuels}
			\subsubsection{Théoriques}
	\section{Sous-espaces vectoriels}
		\subsection{Définition}
		\subsection{Caractérisation}
		\subsection{Exemples}
			\subsubsection{Dans $\mathbb{C}$ : $\mathbb{R}$ev}
			\subsubsection{Dans $\mathbb{R}^2{}$, $\mathbb{R}$ev}
			\subsubsection{...}
		\subsection{Intersection de sous-espaces vectoriels}
			\subsubsection{Démonstration}
			\subsubsection{Remarque}
	\section{Familles libres ou liées}
		\subsection{Définitions}
			\subsubsection{Partie libre}
			\subsubsection{Partie liée}
		\subsection{Propriétés immédiates}
		\subsection{Deux résultats}
	\section{Sous-espace engendré par une partie}
		\subsection{Présentation générale}
			\subsubsection{Définition}
			\subsubsection{Description}
			\subsubsection{Cas d'une partie génératrice $\Omega$ finie}
		\subsection{Somme de sous-$\mathbb{K}$ev}
			\subsubsection{Définition}
			\subsubsection{Propriété}
		\subsection{Sommes directes}
			\subsubsection{Pour deux sous-$\mathbb{K}$ev de $E$}
			\subsubsection{Généralisation à  $n$ sous-$\mathbb{K}$ev de $E$}
		\subsection{Sous-espaces supplémentaires}
			\subsubsection{Définition}
			\subsubsection{Caractérisation}
			\subsubsection{Exemple usuelle}
			\subsubsection{Contre-exemple usuel}
			
\chapter{$\mathbb{K}$-espaces vectoriels de dimension finie}
	\section{Présentation}
		\subsection{Définition}
		\subsection{Comparaison des cardinaux d'une partie libre et d'une partie génératrice}
			\subsubsection{Démonstration}
			\subsubsection{Illustration}
		\subsection{Bases}
			\subsubsection{Définition}
			\subsubsection{Caractérisation}
			\subsubsection{Exemples usuels}
			\subsubsection{Existence de bases}
			\subsubsection{Théorèmes de la base incomplète et de la base extraite}
	\section{Dimension d'un $\mathbb{K}$ev de dimension finie}
		\subsection{Définition}
		\subsection{Exemples usuels}
		\subsection{Caractérisation d'une base quand on connaît la dimension}
			\subsubsection{Énoncé}
			\subsubsection{Démonstration}
			\subsubsection{Illustration}
		\subsection{Formules de dimensions}
			\subsubsection{Inclusion, égalité}
			\subsubsection{Somme}
			\subsubsection{Produit cartésien}
			\subsubsection{Supplémentaire}
	\section{Rang d'un système de vecteurs}
		\subsection{Définition}
		\subsection{Deux résultats pratiques}
			\subsubsection{Système triangulaire}
			\subsubsection{Théorème d'invariance}
		\subsection{Méthode}

\chapter{Applications linéaires}
	\section{Présentation}
		\subsection{Définitions}
			\subsubsection{Générale}
			\subsubsection{Précisions}
			\subsubsection{Compléments}
		\subsection{Exemples usuels}
		\subsection{Propriétés simples}
			\subsubsection{Sous-$\mathbb{K}$evs}
			\subsubsection{Caractérisation de l'injectivité}
			\subsubsection{Composée}
			\subsubsection{Réciproque}
		\subsection{Structures}
	\section{Précisions quand $E$ est de dimension finie}
		\subsection{Concernant Im$(f)$}
		\subsection{Si de plus $(e_1,...,e_n)$ est une base de $E$}
		\subsection{Étude analytique quand $E$ et $F$ sont de dimension finie}
			\subsubsection{Données}
			\subsubsection{But}
			\subsubsection{Résolution}
			\subsubsection{Conclusion}
	\section{Rang d'une AL}
		\subsection{Définition}
		\subsection{Théorème du rang}
		\subsection{Conséquence lorsque dim$(E)$ = dim$(F)$}
		\subsection{Isomorphismes}
			\subsubsection{Caractérisation de la bijectivité d'une AL}
			\subsubsection{Première conséquence}
			\subsubsection{Conséquence générale}
			\subsubsection{Invariance du rang par composition par isomorphisme}
	\section{Projecteurs et symétries}
		\subsection{Définitions}
		\subsection{Propriétés}
			\subsubsection{Relation}
			\subsubsection{Linéarité}
			\subsubsection{Composée}
			\subsubsection{Noyau, image}
		\subsection{Caractérisation}
			\subsubsection{Pour un projecteur}
			\subsubsection{Pour une symétrie}
			\subsubsection{Pratique}
			\subsubsection{Contre-exemple}	
	\section{Formes linéaires et hyperplans}
		\subsection{Formes linéaires}
			\subsubsection{Définition}
			\subsubsection{Exemple usuel}
			\subsubsection{Étude analytique}
		\subsection{Hyperplans en dimension quelconque}
			\subsubsection{Définition générale}
			\subsubsection{Propriété générale}
		\subsection{Hyerplan en dimension finie}
			\subsubsection{Caractérisation}
			\subsubsection{Équation}
			\subsubsection{Intersection d'hyperplans, sens direct}
			\subsubsection{Sens réciproque}

\chapter{Matrices (structures)}
	\section{Définitions}
		\subsection{Générale}
		\subsection{Matrice carrée}
	\section{Opérations}
		\subsection{Combinaison linéaire}
		\subsection{Produit}
			\subsubsection{Condition nécessaire d'existence}
			\subsubsection{Pratique}
			\subsubsection{Expression générale}
			\subsubsection{Exemple}
			\subsubsection{Exemple générale de $AX$}
			\subsubsection{Propriétés diverses du produit de matrices}
			\subsubsection{Transposée d'un produit}
	\section{Structures}
		\subsection{Structure de groupe abélien pour $(M_{np}(\mathbb{K}),+)$}
			\subsubsection{Élément neutre}
			\subsubsection{Symétrique}
		\subsection{Structure d'anneau pour $(M_{n}(\mathbb{K}),+,\times)$}
			\subsubsection{Explication}
			\subsubsection{Compléments}
			\subsubsection{Sous-anneau}
		\subsection{Structure de groupe pour  $(GL_{n}(\mathbb{K}),\times)$}
			\subsubsection{Définition}
			\subsubsection{Structure}
			\subsubsection{Condition suffisante}
			\subsubsection{Inverse d'une transposée}
			\subsubsection{Pratique}
			\subsubsection{Condition nécessaire et suffisante d'inversibilité d'une matrice triangulaire}
			\subsubsection{Sous-structure}
			
\chapter{Matrices et applications linéaires}
	\section{Définitions}
		\subsection{Matrice d'un système de vecteurs}
		\subsection{Matrice d'une application linéaire}
		\subsection{Cas d'un endomorphisme}
		\subsection{Étude analytique}
	\section{Isomorphismes}
		\subsection{Résultat général}
		\subsection{Conséquence}
		\subsection{Isomorphisme transposition}
	\section{Produit matriciel et composition d'AL}
		\subsection{Correspondance}
		\subsection{Isomorphisme d'anneaux}
		\subsection{Conséquence}
	\section{Changements de base}
		\subsection{Matrices de changement de base}
			\subsubsection{Définition}
			\subsubsection{Caractérisation}
			\subsubsection{Étude analytique}
		\subsection{Effet sur une AL}
			\subsubsection{Données}
			\subsubsection{Calculs}
			\subsubsection{Conclusion}
		\subsection{Cas d'un endomorphisme}
		\subsection{Matrices équivalentes}
		\subsection{Matrices semblables}
	\section{Rang d'une matrice}
		\subsection{Définition}
		\subsection{Caractérisation}
		\subsection{Application}
	\section{Trace de matrice carrée}
		\subsection{Définition}
		\subsection{Propriétés}
				
\chapter{Applications multilinéaires}
	\section{Présentation}
		\subsection{Définition}
		\subsection{Exemples usuels}
		\subsection{Compléments de définition}
		\subsection{Propriétés simples des formes $n$-linéaires antisymétriques (ou alternées)}
	\section{Cas des formes $n$-linéaires antisymétriques sur $E_n$}
		\subsection{Étude analytique}
		\subsection{Rappel}
		\subsection{Conséquence de structure}
		\subsection{Condition nécessaire et suffisante d'indépendance linéaire}

\chapter{Déterminants}
	\section{Présentation}
		\subsection{Déterminant de $n$ vecteurs dans un $\mathbb{K}$ev de dimension $n$}
		
\chapter{Systèmes d'équations linéaires}
	\section{Présentation}
		\subsection{Définition}
		\subsection{Traductions}
		\subsection{Traductions}
		\subsection{Structure de l'ensemble des solutions}
	\section{Méthode de résolution}
	\section{Exemples}
		\subsection{Cas où $n=p$ (=3)}
		\subsection{Cas où $n>p$}
		\subsection{Cas où $n<p$}
		\subsection{Cas où $n=p$ (quelconque)}
		\subsection{Autre exemple : classiquement par le pivot de Gauss}
		\subsection{Exemple résolu "en rusant"}
		
\chapter{Complément : opérations élémentaires sur les lignes et les colonnes d'une matrice}
	\section{Présentation}
	\section{Traduction matricielle}
	\section{Principe pour inverser une matrice}
	\section{Exemple}

	
	
\end{document}