\documentclass[12pt,a4paper,french]{book}
\usepackage{graphicx} % Required for inserting images
\usepackage{multirow}
\usepackage[frenchb]{babel}
\usepackage{fancybox,framed}
\usepackage{amssymb}
\usepackage{amsmath}
\usepackage{array}
\title{Cours}
\author{C. LACOUTURE}
\date{Année scolaire 2024-2025, MPSI2, Lycée Carnot}
\begin{document}
	
\maketitle
\tableofcontents
\part{Arithmétique}
\chapter{Récurrences, sommations}
\section{Récurrences}
\subsection{Principe général}
\subsubsection{Axiome}
\ovalbox{si} $A$ est une partie de $\mathbb{N}$ tq $\left\{ 
\begin{array}{ll}
	0 \in A\\ 
	\forall n \in A, n+1 \in A
\end{array}
\right. $ \ovalbox{alors} $A = \mathbb{N}$
\subsubsection{Énoncé}
\begin{framed} Soit $\mathcal{P}(n)$ une propriété dépendante de $n \in N$.
	\ovalbox{si} $\left\{ 
	\begin{array}{ll}
		\mathcal{P}(0) \mbox{vraie}\\ 
		\forall n \in \mathbb{N}, \mathcal{P}(n) \Rightarrow \mathcal{P}(n+1)
	\end{array}
	\right. $ \ovalbox{alors} $\mathcal{P}(n)$ est vraie $\forall n \in \mathbb{N}$
\end{framed}
\subsubsection{Applications à des expressions de sommes usuelles}
\begin{itemize}
	\item Somme d'entiers consécutifs :
	\[ \sum_{k=0}^{n} k = 0 + 1 + 2 + ... + n = \frac{n(n+1)}{2}\]
	\item Somme des carrés :
	\[ \sum_{k=0}^{n} k^{2} = 0^{2} + 1^{2} + 2^{2} + ... + n^{2} = \frac{n(n+1)(2n+1)}{6}\]
	
	Démonstration : soit $\mathcal{P}(n)$ l'égalité ci-dessus. 
	\begin{itemize}
		\item $\mathcal{P}(0)$ est vraie car $\sum_{k=0}^{0} k^{2} = 0 = \frac{0(0+1)(2\cdot0+1)}{6}$
		\item \ovalbox{si} pour un certain $n \in \mathbb{N}$, $\mathcal{P}(n)$ est vraie \ovalbox{alors} montrons que $\mathcal{P}(n+1)$ est vraie. En effet :
		\begin{equation} 
			\begin{split}
				\sum_{k=0}^{n+1} k^{2} &= \sum_{k=0}^{n} k^{2} + (n+1)^{2} \\ &= \frac{n(n+1)(2n+1)}{6} + (n+1)^{2} \mbox{  par hypothèse de récurrence} \\ &= \frac{n(n+1)(2n+1)+6(n+1)^{2}}{6} \\ &= \frac{(n+1)(n(2n+1)+6(n+1))}{6} \\ &= \frac{(n+1)(2n^{2}+7n+6)}{6} \\ &= \frac{(n+1)(n+2)(2n+3)}{6} \\&= \frac{(n+1)((n+1)+1)(2(n+1)+1)}{6}
			\end{split}
		\end{equation}
		donc $\mathcal{P}(n+1)$ est bien vraie.
	\end{itemize}
	\item Sommes des cubes : 
	\[ \sum_{k=0}^{n} k^{3} = 0^{3} + 1^{3} + 2^{3} + ... + n^{3} = \left( \frac{n(n+1)}{2}\right)^{2} = \left( \sum_{k=0}^{n} k\right)^{2} \]
	\item Somme géométrique : soit $a \in \mathbb{R}$
	\[ \sum_{k=0}^{n} a^{k} = 1 + a^{1} + a^{2} + ... + a^{n} = \left\{ 
	\begin{array}{ll}
		\frac{1-a^{n+1}}{1-a} &\mbox{ si } a \neq 1 \\ 
		n+1 &\mbox{ si } a=1
	\end{array}
	\right.
	\]
\end{itemize}
		\subsection{Utilisation la plus fréquente}
		\subsection{Variantes}
		\subsection{Originalités}
	\section{Sommations}
		\subsection{Somme simple}
		\subsection{Sommes doubles}
		
\chapter{Coefficients binomiaux}
	\section{Définitions}
		\subsection{Factorielle}
		\subsection{Coefficient binomial}
			\subsubsection{Générale}
			\subsubsection{Cas particulier}
	\section{Formules du triangle de Pascal}
		\subsection{Énoncé}
		\subsection{Pratique}
		\subsection{Expression des coefficients binomiaux}
	\section{Applications}
		\subsection{Symétrie des coefficients binomiaux}
		\subsection{Formule du binôme de Newton}
		\subsection{Linéarisation}
		\subsection{Inversement}
		\subsection{Nombre de parties d'un ensemble}
		
\chapter{PGCD, PPCM, nombres premiers}
	\section{PPCM : Plus Petit Commun Multiple}
		\subsection{Définition}
		\subsection{Propriétés}
		\subsection{Cas particulier}
	\section{PGCD : Plus Grand Commun Diviseur}
		\subsection{Définition}
		\subsection{Propriétés}
		\subsection{Cas particulier}
		\subsection{Pratique}
			\subsubsection{Résultat préliminaire}
			\subsubsection{Algorithme}
			\subsubsection{Notations}
			\subsubsection{Concrètement}
			\subsubsection{Exemple : pgcd(56,23)}
			\subsubsection{Remarque}
	\section{Nombres premiers entre eux}
		\subsection{Définition}
		\subsection{Caractérisation : théorème de Bézout}
		\subsection{Propriétés}
	\section{Généralisation}
		\subsection{pgcd,ppcm de $n$ entiers}
			\subsubsection{ppcm}
			\subsubsection{pgcd}
		\subsection{Pratique}
	\section{Nombres premiers}
		\subsection{Définition}
		\subsection{Premières propriétés}
		\subsection{Infinité}
			\subsubsection{Résultat préliminaire}
			\subsubsection{Conséquence}
		\subsection{Crible d'Eratosthène}
			\subsubsection{Résultat préliminaire}
			\subsubsection{Principe du crible}
			\subsubsection{Exemple}
		\subsection{Décomposition en facteurs premiers}
			\subsubsection{Énoncé}
			\subsubsection{Démonstration}
	\section{Congruences}
		\subsection{Définition de la relation}
			\subsubsection{$\equiv (n)$ est une relation d'équivalence}
			\subsubsection{Classes d'équivalence}
		\subsection{Opérations sur $\mathbb{Z}/ n\mathbb{Z}$}
			\subsubsection{Addition}
			\subsubsection{Multiplication}
			\subsubsection{Justification de la définition}
		\subsection{Structures}
		
\chapter{Dénombrements}
	\section{Cardinal d'un ensemble fini}
		\subsection{Définitions}
		\subsection{Propriétés}
			\subsubsection{Réunion disjointe}
			\subsubsection{Sous-ensemble}
			\subsubsection{Réunion quelconque}
			\subsubsection{Produit cartésien}
	\section{Applications entre deux ensembles finis}
		\subsection{Remarque préliminaire}
		\subsection{Dénombrement}
		\subsection{Quand $E$ et $F$ sont de même cardinal (fini)}
	\section{Parties d'un ensemble fini}
		\subsection{$p$-liste d'éléments distincts}
		\subsection{Autre démonstration de card$(\mathcal{P}(E)) = 2^{\mbox{card}(E)}$}
		\subsection{Retrouvons que $\begin{pmatrix}
				n \\ p
			\end{pmatrix} = \frac{n!}{p!(n-p)!}$}
	
	
	
\end{document}